\setdetailtitle{OpenCRS: Attack Surface Approximation, Vulnerabilities Discovery, and Automatic Exploitation of Binaries}

\setdetailsubtitle{}

\setdetailstudentname{George-Andrei Iosif}

\setdetailadvisor{Adrian-Răzvan Deaconescu}

\setdetailsecondaryadvisor{Constantin-Eduard Stăniloiu}

\setdetailyear{2023}

\setdetailabstractromanian{Obiectivul acestei teze este de a introduce patru module ale OpenCRS, un nou sistem de raționament cibernetic (CRS) cu sursă deschisă. Un CRS este o colecție de componente software proiectate pentru a identifica, exploata și mitiga vulnerabilități, totul într-o manieră automatizată. În ciuda trecerii a șapte ani de la Cyber Grand Challenge, o competiție organizată de DARPA ce a introdus această familie de sisteme de securitate, contextul CRS-urilor cu sursă deschisă a rămas neschimbat, cu finalistul Shellphish considerat încă cel mai evoluat. Această teză introduce module ce încorporează avansul tehnologic realizat din 2016 până în prezent. Aceste module includ o colecție de programe vulnerabile, un modul pentru aproximarea suprafeței de atac a executabilelor, un modul de detecție a vulnerabilităților folosind fuzzing și un modul de exploatare automată. Teza va prezenta metodic fiecare modul prin definirea tehnologiilor implicate, prezentarea arhitecturii software și a funcționalității, cât și validarea rezultatelor prin teste și evaluări.}

\setdetailabstractenglish{The objective of this thesis is to introduce four distinct modules of OpenCRS. The latter is a novel open source cyber reasoning system (CRS), which is an automated collection of software components designed to identify, exploit, and patch vulnerabilities in an automated fashion. Despite the passage of seven years since the completion of Cyber Grand Challenge, a competition organized by DARPA that introduced this family of automated security systems, the open source CRS landscape has exhibited a lack of change. The finalist Shellphish continues to be regarded as the prevailing state-of-the-art solution in this domain. The thesis introduces modules that incorporate the technological advancements achieved between 2016 and the present time. These modules include a dataset containing public test suites of vulnerable programs, a module designed to approximate the attack surface of an executable, a vulnerability detection module that utilizes fuzzing techniques, and a final module focused on automated exploitation. The approach taken in each module will be methodical, involving the definition of employed technologies, an in-depth explanation of the software architecture and its functioning, and the presentation of tests and evaluations.}

\setdetailacknowledgements{I would like to thank everyone who contributed to the creation of this thesis.

    Adrian-Răzvan Deaconescu and Constantin-Eduard Stăniloiu, the thesis
    coordinators, were available for regular support meetings, which helped me and
    other colleagues working on OpenCRS define our theses and the open source
    project.

    This thesis' peer reviews were provided by Andreia-Irina Ocănoaia and Amir
    Naseredini. Their suggestions were extremely helpful and most of them have been
    incorporated into the current version of the text.

    Finally, much like during the bachelor's studies, Iulian was available to offer
    helpful guidance based on his personal experience.}
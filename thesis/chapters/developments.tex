\documentclass[../main.tex]{subfiles}
\graphicspath{{\subfix{../images/}}}

\begin{document}

\hypertarget{further-development}{%
\chapter{Further Development}\label{further-development}}

Due to the goal of creating a proof of concept for an open-source cyber
reasoning system, we imposed the \textbf{limitations} presented in the
chapter \ref{opencrs} such that our implementation could reach the desired
maturity and wholeness.

Firstly, the possible future improvements may target the
fundamental way in which OpenCRS works:

\begin{itemize}
\tightlist
\item
  New executable formats: PE, Mach-O;
\item
  New CPU architectures: \texttt{x86-64}, \texttt{amd64};
\item
  New input streams: network packets;
\item
  New programming languages
\item
  Unification of all modules' configuration into one file.
\end{itemize}

Besides this, the changes may improve the implemented module.

\hypertarget{dataset-module}{%
\section{Dataset Module}\label{dataset-module}}

\begin{itemize}
\tightlist
\item
  Static code analysis over the source files included in the test
  suites, such that the input streams are detected and used as labels in
  the dataset
\item
  More tests in the suite created by us by importing the Zeratool's
  binaries and creating small and handmade executables (which can be
  used in the testing phases because of the execution speed)
\item
  More test suits, such as \texttt{cb-multios}\simplefootnote{https://github.com/trailofbits/cb-multios}, a Linux port of
  the executables used in DARPA's Cyber Grand Challenge
\end{itemize}

\hypertarget{attack-surface-approximation-module}{%
\section{Attack Surface Approximation
Module}\label{attack-surface-approximation-module}}

\begin{itemize}
\tightlist
\item
  Improve the accuracy of the binary matching heuristic by searching
  only in specific sections of the executable
\item
  Improve the accuracy of arguments fuzzing by studying other execution
  events available in QBDI
\item
  New technique to detect the usage of an input stream, such as
  symbolically running the binary and intercepting the input-related
  (library or system) calls
\item
  Pair-wise testing of arguments, for scenarios of dependency relations
\item
  Migration from mounted Docker volumes to gRPC for the QBDI and Ghidra
  containers
\end{itemize}

\hypertarget{vulnerability-discovery-module}{%
\section{Vulnerability Discovery
Module}\label{vulnerability-discovery-module}}

\begin{itemize}
\tightlist
\item
  Timeout or coverage convergence heuristic for stopping the fuzzing
  session, for example by leveraging Fuzz Introspector\simplefootnote{https://github.com/ossf/fuzz-introspector}
\item
  Binary rewriting for adding Address Sanitizer, for example with
  RetroWrite\simplefootnote{https://github.com/HexHive/retrowrite}
\item
  Migration from mounted Docker volumes to gRPC for the afl++ containers
\end{itemize}

\hypertarget{automatic-exploit-generation}{%
\section{Automatic Exploit
Generation}\label{automatic-exploit-generation}}

\begin{itemize}
\tightlist
\item
  Reverification of the created exploit in a sandboxed environment
\item
  Exploitation via format string attacks, a technique that is
  implemented by the already-integrated \texttt{zeratool\_lib}
\item
  New exploitation techniques, eventually for other input streams too
\end{itemize}

\end{document}
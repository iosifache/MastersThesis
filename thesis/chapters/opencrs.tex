\documentclass[../main.tex]{subfiles}
\graphicspath{{\subfix{../images/}}}

\begin{document}

\hypertarget{opencrs}{%
\chapter{OpenCRS}\label{opencrs}}

This study, together with Claudiu Ghenea's \cite{ghenea}, presents an open-source
cyber reasoning system. OpenCRS aims to implement a comprehensive
security assessment process for executables by utilizing the latest
advancements in binary analysis from both academic and industrial
sectors. The purpose of OpenCRS is to execute a comprehensive security
evaluation of executable files, encompassing the identification of input
methods and the development of patches to address any detected
vulnerabilities.

Given the significant diversity in architectures, programming languages,
and binary behaviors, the objective was to create a proof of concept
that addresses the execution of programs exhibiting the subsequent
attributes:

\begin{itemize}
\tightlist
\item
  The system operates on a 32-bit Intel architecture, specifically
  utilizing the \texttt{i386} instruction set architecture.
\item
  The software is compatible with the Linux operating system and
  utilizes the ELF format.
\item
  These are generated from the source code written in the C programming
  language.
\item
  The input streams for arguments, standard input, and files are
  utilized.
\end{itemize}

The diagram below depicts the comprehensive architecture of OpenCRS and
the inter-module communication taking place within the system.

\begin{landscape}
\renewcommand*\figurename{Figure}
\begin{figure}[!h]
   \centering
    \includegraphics[height=0.95\textheight]{images/opencrs.png}
    \caption{OpenCRS's Architecture}
    \label{fig:opencrs_architecture}
\end{figure}
\end{landscape}

Given that an analyst demands an executable analysis, the orchestration
module manages the modules and their internal procedures as follows:

\begin{enumerate}
\def\labelenumi{\arabic{enumi}.}
\tightlist
\item
  Dataset module: By incorporating various public test suites into the C
  source code, the module can compile it and generate a sequence of
  executable files. The primary benefit of this methodology is that
  OpenCRS can implement a validation mechanism that takes into account
  the vulnerabilities present in the source code, as indicated by the
  CWE labels, as well as those that have been identified, exploited, and
  remediated by the system. The dataset module is not mandatory since
  the analyzed executables utilized by OpenCRS may have already been
  constructed from alternative origins. Both open-source software, such
  as HiColor\simplefootnote{https://github.com/dbohdan/hicolor}, and closed-source software, such as Dropbox\simplefootnote{https://dropbox.com}, offer their
  users the option to download prebuilt binaries that cater to a variety
  of architectures.
\item
  Attack surface approximation module: With an executable (and no other
  knowledge about it) as input, this module will determine how it might
  be attacked: either through input streams or a predefined format for
  them (for example, the arguments that the program expects in
  \texttt{argv}).
\item
  Vulnerability discovery module: Subsequent to the identification of
  the attack surface by the preceding module, this module will employ
  vulnerability discovery methodologies such as fuzzing and symbolic
  execution to ascertain the existence of vulnerabilities, specifically
  inputs that result in erroneous program behavior.
\item
  Vulnerability analytics module: Upon detecting a proof of
  vulnerability, this module conducts an analysis to provide additional
  information regarding the specific vulnerability that has been
  identified. This may include identifying the category of
  vulnerability, such as a buffer overflow.
\item
  Automatic exploit generation module: The preceding module's proof of
  vulnerability is utilized to develop an exploit that triggers it and
  has the greatest possible impact.
\item
  Signature generation module: In contrast to the offensive nature
  of its predecessor, the module responsible for generating signatures
  serves a protective function. By utilizing identical input as the
  automatic exploit generation module, a signature is generated with the
  purpose of identifying and preventing exploitation endeavors. The
  module is deemed valuable in scenarios involving critical or outdated
  systems, wherein the installation of an alternative program is not
  feasible due to the unavailability of the program or compatibility
  issues.
\item
  Healing module: The objective of this module is akin to that of the
  signature generation module, which is to safeguard the binary from
  exploitation techniques. In contrast to the preceding module, the
  current one has made alterations to the binary in order to eliminate
  the vulnerable code or implement appropriate sanitization measures,
  while preserving the original functionality.
\end{enumerate}

Subsequent chapters will provide a comprehensive account of the internal
operations of various modules. It is noteworthy that the outstanding
modules have been addressed in separate theses: the vulnerability
analytics and healing modules in \cite{ghenea}, and the signature generation
module in \cite{stefan}.

\end{document}
\documentclass[../main.tex]{subfiles}
\graphicspath{{\subfix{../images/}}}

\begin{document}

\hypertarget{testing-and-assesment}{%
  \chapter{Testing and Assessment}\label{testing-and-assesment}}

The forthcoming chapter will delineate the process of functional testing and
evaluation of the OpenCRS modules that were previously explicated.

For testing and evaluation, a virtual machine was employed, which was
configured to run Ubuntu 22.04. The guidelines for configuring each module were
followed, thus guaranteeing the fulfillment of all requirements, including
Python libraries, Docker, and built images.

Even though every module possesses the ability to operate as a Python library,
we have chosen to employ the command line interface of each module owing to its
superior suitability for manual testing. The integration of the modules into
the OpenCRS will involve the utilization of a Python library within an
orchestration module.

\hypertarget{dataset-generation}{%
  \section{Dataset Generation}\label{dataset-generation}}

For the dataset module, we have constructed all three integrated test suites,
which are identifiable in the OpenCRS context by a unique ID:
\texttt{nist\_juliet}, \texttt{nist\_c\_test\_suite}, and
\texttt{toy\_test\_suite}. The present task involved the utilization of the
\texttt{built} command in combination with the \texttt{get} command, which
serves to retrieve the index of the built executables.

\begin{tiny}
\begin{verbatim}
$ opencrs-dataset build --testsuite NIST_C_TEST_SUITE
Successfully built 50 executables.
$ opencrs-dataset get
The available executables are:

-----------------------------------------------------------------------------------------------------------------------------------
| ID                    | CWEs                                       | Parent Database    | Full Path                             |
|-----------------------|--------------------------------------------|--------------------|---------------------------------------|
| nist_c_test_suite_113 | Use of Externally-Controlled Format String | nist_c_test_suite  | executables/nist_c_test_suite_113.elf |
| nist_c_test_suite_133 | Use of Externally-Controlled Format String | nist_c_test_suite  | executables/nist_c_test_suite_133.elf |
[...]
\end{verbatim}
\end{tiny}

The outcome of this procedure yielded a set of 54591 vulnerable ELF
executables, which were generated from C source code and designed for the
32-bit i386 architecture. The open-source community was granted access to the
dataset through the establishment of a distinct repository, namely
\texttt{opencrs\_dataset} \cite{opencrs_dataset}.

The observed distribution of executables across the test suites, namely
\texttt{54531} for \texttt{nist\_juliet}, \texttt{50} for
\texttt{nist\_c\_test\_suite}, and \texttt{10} for \texttt{toy\_test\_suite},
suggests the presence of compilation errors during the executables' generation
process. Upon analyzing the build logs, it has been determined that the primary
factors contributing to the issue are the absence of essential headers such as
\texttt{mysql/mysql.h} and \texttt{security/pam\_appl.h}, as well as
inconsistencies between the Windows and Linux operating systems, such as the
utilization of data types such as \texttt{wchar\_t}.

The executable details were recorded in a CSV file denominated as
\texttt{index.csv}, featuring the subsequent labels:

\begin{itemize}
  \tightlist
  \item
        \texttt{name}: Unique identifier of a program, the format
        \texttt{executables/\textless{}name\textgreater{}.elf} being utilized to
        determine the path of the executable file;
  \item
        \texttt{cwes}: One or more CWEs that are present in the executable;
        and
  \item
        \texttt{parent\_dataset}: Parent dataset's identifier.
\end{itemize}

Finally, it is noteworthy that the employed test suites do not conform to a
consistent distribution of executables across each vulnerability type, as
identified by the CWE ID.

\begin{table}[]
\centering
\begin{tabular}{|l|l|}
\hline
Weakness                                   & Count \\ \hline
Stack-based Buffer Overflow                & 13836 \\ \hline
Heap-based Buffer Overflow                 & 11088 \\ \hline
Integer Overflow or Wraparound             & 3960  \\ \hline
Mismatched Memory Management Routines      & 3564  \\ \hline
Integer Underflow                          & 2952  \\ \hline
Free of Memory not on the Heap             & 2680  \\ \hline
Use of Externally-Controlled Format String & 2410  \\ \hline
Buffer Underflow                           & 2048  \\ \hline
Buffer Under-read                          & 2048  \\ \hline
OS Command Injection                       & 1921  \\ \hline
\end{tabular}
\caption{\label{executables-distribution}Distribution of Vulnerable Executables, by CWE}
\end{table}

\hypertarget{arguments-dictionaries-generation}{%
  \section{Arguments Dictionaries
    Generation}\label{arguments-dictionaries-generation}}

Subsequently, the module subjected to testing and evaluation was the attack
surface discovery module.

Through utilization of the \texttt{generate} command and the three implemented
heuristics, it is possible to generate dictionaries of arguments that may be
employed for the purpose of argument fuzzing:

\begin{itemize}
  \tightlist
  \item
        \texttt{generation}: \texttt{62} entries with arguments like
        \texttt{-a}, \texttt{-D}, and \texttt{-9};
  \item
        \texttt{man\_parsing}: \texttt{6701} entries with arguments like
        \texttt{-\/-ascii}, \texttt{-\/-cert-status}, and \\
        \texttt{-\/-message-id}; and
  \item
        \texttt{binary\_pattern\_matching} for \texttt{/usr/bin/uname}:
        \texttt{37} entries with arguments like \\ \texttt{-\/-processor}, \texttt{-\/-operating-system}, and \texttt{-linux-x86-64}. As evident from the
        list, the precision is not unitary since not all the strings
        returned are valid arguments. This phenomenon occurs due to the
        adherence of certain character sequences within the binary files to
        the regular expression pattern for arguments. However, this does not
        present an impediment, as the aforementioned arguments will undergo
        two additional filtering stages, specifically the argument fuzzing and
        the vulnerability detection module.
\end{itemize}

\begin{tiny}
\begin{verbatim}
$ opencrs-surface generate --heuristic binary_pattern_matching --elf /usr/bin/uname --output uname.dict
Successfully generated dictionary with 37 arguments
$ head -3 uname.dict
--operating-system
--processor
--version
\end{verbatim}
\end{tiny}

\hypertarget{arguments-fuzzing}{%
  \section{Arguments Fuzzing}\label{arguments-fuzzing}}

The \texttt{fuzz} command was subjected to testing and evaluation in the
identical module, in comparison to the \texttt{uname} Linux utility, which is
utilized for obtaining details regarding the present host and operating system.
The previously generated dictionary was utilized for reference purposes.

\begin{tiny}
\begin{verbatim}
$ opencrs-surface fuzz --elf /usr/bin/uname --dictionary uname.dict
Several arguments were detected for the given program:

-----------------------------------------------
| Argument                   |      Role      |
|----------------------------|----------------|
| -                          |      FLAG      |
| --all                      |      FLAG      |
| --all string               |      STRING      |
| --hardware-platform        |      FLAG      |
| --hardware-platform string | STRING_ENABLER |
[...]
\end{verbatim}
\end{tiny}

\texttt{-}, \texttt{-\/-all}, \texttt{-\/-hardware-platform},
\texttt{-\/-kernel-name}, \texttt{-\/-kernel-release}, \texttt{-\/-machine}, \\ \texttt{-\/-kernel-version}, 
\texttt{-\/-nodename}, \texttt{-\/-operating-system}, and
\texttt{-\/-processor} were identified as valid program arguments with
two roles: \texttt{FLAG} (valid since they activate functionality) and
\texttt{STRING\_ENABLER} (invalid because they do not require the user
to give any string after that). Upon examining the outcomes produced by
QBDI, it can be inferred that the reason for this occurrence is due to
the fact that the DJB2 hash generated for \texttt{<param>}
(e.g. ~\texttt{-939574273} for \texttt{-a}) and for
\texttt{<param>\ \textless{}string\textgreater{}} (e.g. ~\texttt{657648750} for \texttt{-a <string>})
are different.

Conversely, the hyphen symbol (\texttt{-}) does not constitute a legitimate
argument, and the argument \texttt{-\/-version}, which was originally
incorporated in the produced dictionary, was not identified as a valid one. The
observed occurrence could be attributed to the implementation of
\texttt{uname}, as QBDI fails to recognize the code flow related to this option
as a basic block transfer.

The final aspect to note regarding the functionality of the attack surface
approximation module pertains to the detection of aliases. The \texttt{uname}
command has the capability to accept abbreviated forms for all of its preceding
arguments. By way of illustration using the \texttt{-\/-all} option and further
scrutinizing the QBDI outcomes, it is observable that the hash value of
\texttt{-939574273} is obtained when the flag role is tested, corresponding to
the DJB2 hash of the \texttt{-a} parameter. Upon collision detection by the
OpenCRS module, it will selectively incorporate solely the initial argument
from the dictionary that has been detected with the corresponding hash (namely,
\texttt{-\/-all}).

\hypertarget{input-streams-detections}{%
  \section{Input Streams Detections}\label{input-streams-detections}}

Four executables from the toy test suite were utilized to evaluate the efficacy
of input stream detection. Each of the aforementioned programs possessed a
minimum of one input stream originating from a collection consisting of
\texttt{stdin}, files, and arguments.

\begin{tiny}
\begin{verbatim}
$ opencrs-surface detect --elf null_pointer_deref_args.elf
Several input mechanisms were detected for the given program:

----------------------------------
| Stream               | Present |
|----------------------|---------|
| STDIN                |   No    |
| ARGUMENTS            |   Yes   |
| FILES                |   No    |
| ENVIRONMENT_VARIABLE |   No    |
| NETWORKING           |   No    |
----------------------------------
\end{verbatim}
\end{tiny}

Table \ref{executables-distribution} presents a summary of the outcomes.

\begin{table}[]
\centering
\begin{tabular}{|l|l|l|l|}
\hline
Executable                      & Arguments Stream & Files Stream  & stdin Stream  \\ \hline
\texttt{null\_pointer\_deref\_args.elf}  & Detected (TP)    & N/A           & N/A           \\ \hline
\texttt{null\_pointer\_deref\_files.elf} & Detected (TP)    & Detected (TP) & Detected (FP) \\ \hline
\texttt{null\_pointer\_deref\_stdin.elf} & Detected (TP)    & Detected (FP) & N/A           \\ \hline
\texttt{multiple\_inputs\_streams.elf}   & Detected (TP)    & Detected (TP) & Detected (TP) \\ \hline
\end{tabular}
\caption{\label{executables-distribution}Accuracy in Detecting the Input Streams}
\end{table}

In two programs, the file and \texttt{stdin} streams were wrongly recognized as
present. This phenomenon occurs due to the utilization of reading library calls
by the programs, which lack specificity towards any particular one. It is
noteworthy that these entities possess the capability to read from a file
pointer, thereby enabling them to manage input from both \texttt{stdin} (the
\texttt{0} descriptor) and files (the descriptor obtained through invocations
such as \texttt{open} and \texttt{fopen}). For this reason, the modules tend to
prioritize the promotion of false positives (streams that are not actually
genuine but can be confirmed through the vulnerability detection module) over
false negatives (streams that are authentic but remain unreported).

\hypertarget{vulnerability-detection}{%
  \section{Vulnerability Detection}\label{vulnerability-detection}}

To identify vulnerabilities using the three implemented fuzzers, we selected
three programs that experienced crashes due to distinct causes when provided
with a particular input on supported input streams. The identified weaknesses
or susceptibilities were:

\begin{itemize}
  \tightlist
  \item
        Tainted format string: If \texttt{\%s} was used as an input format
        string, a \texttt{NULL} pointer dereferencing occurred. Another attack
        used \texttt{\%n} to try to write to the same \texttt{NULL} pointer.
  \item
        Stack buffer overflow: If the input is long enough, the saved EBP and
        EIP registers on the stack are overwritten. The program will crash
        when it returns because the values are invalid.
  \item
        NULL pointer dereferencing.
\end{itemize}

\begin{tiny}
\begin{verbatim}
$ opencrs-vulnerability fuzz --fuzzer FILES_AFLPLUSPLUS --stream FILES --elf tainted_format_string_files.elf
New proof of vulnerability was generated with the following payloads:
- For FILES:

00000000: 6E 25 6E 25                                       n%n%
\end{verbatim}
\end{tiny}

The input supplied by afl++, namely the \texttt{\%n}, prompted the NULL pointer
dereferencing and subsequent crash in this example. The fuzzer successfully
created the crashing inputs in the other two cases, notably a long input for
the executable prone to buffer overflow and another with \texttt{y} on the
fifth place for the one vulnerable to NULL pointer dereferencing.

\hypertarget{automatic-exploit-generation}{%
  \section{Automatic Exploit
    Generation}\label{automatic-exploit-generation}}

The automatic exploit generating module was tested for the final functionality
with three binary given by Zeratool, all having \texttt{stdin} as an input
stream and vulnerable to buffer overflow, but with various mitigations:

\begin{enumerate}
  \def\labelenumi{\arabic{enumi}.}
  \tightlist
  \item
        Two with NX, canaries, and PIE disabled; and
  \item
        One with NX enabled + canaries and PIE disabled.
\end{enumerate}

Furthermore, one executable contained a sensitive function that could be used
in the exploit (as a substitute to, say, other symbols). In every example, the
module effectively developed exploits.

\begin{tiny}
\begin{verbatim}
$ opencrs-aeg exploit --exploiter ZERATOOL --elf bof_nx_32 --stream STDIN --mitigation NX --weakness STACK_OUT_OF_BOUND_WRITE
The exploiter could generate an exploit with the outcome of SHELL and the following payloads:

- For STDIN:
[...]
\x00\x00\x00\x00\x00\x00\x00\x00\x90\x90\x04\x08baaa\x1e\xa0\x04\x08\x00\x00\x00\x00\x00
[...]
\end{verbatim}
\end{tiny}

\end{document}
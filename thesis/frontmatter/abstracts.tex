\newcommand{\AbstractRO}{Anul 2016 a marcat debutul unei competiții fără precedent, și anume Cyber Grand Challenge organizat de DARPA, unde sisteme de raționament cibernetic s-au întrecut într-un concurs de identificare, exploatare și mitigare de vulnerabilități, totul într-o manieră  automatizată. În ciuda trecerii a șapte ani, contextul CRS-urilor cu sursă deschisă a rămas neschimbat, cu finalistul Shellphish considerat încă cel mai evoluat. Obiectivele acestei teze este de a introduce patru module distincte ale unui nou sistem de raționament cibernetic numit OpenCRS. Aceste module includ o colecție de programe vulnerabile, un modul pentru aproximarea suprafeței de atac a executabilelor, un modul de detecție a vulnerabilităților folosind fuzzing și un modul de exploatare automată. Teza va prezenta metodic fiecare modul prin definirea tehnologiilor implicate, prezentarea arhitecturii software și a funcționalității, cât și validarea rezultatelor prin teste și evaluări.}

\newcommand{\AbstractEN}{The year 2016 marked the debut of an unprecedented competition, namely the Cyber Grand Challenge organized by DARPA, wherein cyber reasoning systems engaged in a contest of identifying, exploiting, and patching vulnerabilities in an automated manner. Despite the passage of seven years, the open source CRS landscape has remained unchanged, with the finalist Shellphish still being considered the current state-of-the-art. The objective of this thesis is to introduce four distinct modules of a novel open source cyber reasoning system, referred to as OpenCRS. These modules include a dataset with public test suites of vulnerable programs, a module for approximating the attack surface of an executable, a vulnerability detection module utilizing fuzzing, and a final module for automated exploitation. The thesis will adopt a methodical approach to each module by defining the employed technologies, expounding on the software architecture and its functioning, and illustrating tests and evaluations.}

\begin{titlepage}
  \textbf{\large SINOPSIS}\par
  \AbstractRO\par\vfill
  \textbf{\large ABSTRACT}\par
  \AbstractEN \vfill
\end{titlepage}